\documentclass{article}

\usepackage{amsmath, amssymb}

\newcommand{\R}{\mathbb R}
\newcommand{\Z}{\mathbb Z}

\title{Simplicial Homology and Cohomology\\ A Geometrical Introduction}
\author{Sebastián Sánchez}

\begin{document}

\maketitle

\section{Introduction}

Most of the rigorous definition are from~\cite{munkres}.

\section{Simplicial Complexes}

A simplicial complex is a generalization of a triangulation. The role of triangles is
played by the simplices. A \(d\)-simplex is collection of \(d+1\) affinely independent points
in \(\R^{d}\), affinely indepedent means that the relative vectors are linearly independent:
\begin{displaymath}
  a_0, \dots, a_d \quad\textrm{affinely independent}
  \iff
  a_1-a_0, \dots, a_d-a_0 \quad\textrm{linearly independent}.
\end{displaymath}
Naturally, we can associate a geometric realization of a simplex by taking is convex hull.
A simplicial complex \(K\) is a collection of simplices closed under the relation of subsets and 
closed under the operation of intersection. Explicitely,
\begin{enumerate}
  \item for any \(P\in K\), \(Q\in K\) for any \(Q\subset P\),
  \item for any \(P, S\in K\), \(P\cap S \in K\).
\end{enumerate}

\section{Homology}

As in triangulations and more generally in graphs, we can define a notion of a walk
in a simplicial complex. In order to define a walk, we need a notion of orientation.
This is simply given by describing a simplex in an ordered list (choosen by you or given):
\begin{displaymath}
  P = \lbrace p_0, \ldots, p_d \rbrace 
  \rightarrow
  \langle P \rangle = \langle p_0, \ldots, p_d \rangle.
\end{displaymath}
We denote by \(\langle p_0, \ldots, \hat p_i, \ldots, p_d \rangle\) for the oriented
simplex with the \(p_i\) element removed and \(\langle K \rangle\) for the oriented simplicial
complex. 
Once an orientation is given for all simplices
in the complex, we define a walk as a collection of simplices of the same dimension.
Equivalently, a walk correspond to a function \(c\colon \langle K^{(d)} \rangle \to \Z_2\). Denote
by \(c_{\langle P \rangle}\) the oriented indicator function of the simplex \(\langle P \rangle\), that is:
\begin{displaymath}
  c_{\langle P\rangle}(\langle S\rangle) = 
  \left\{
  \begin{aligned}
    1 &\quad P = S \land \langle P \rangle = \langle S \rangle\\
    -1 &\quad  P=S \land \langle P \rangle \ne \langle S \rangle\\
    0 &\quad P\ne S
  \end{aligned}
  \right.
\end{displaymath}
For easy of notation, we drop the \(\langle \cdot \rangle\), but keep in mind that all simplices
are oriented. Then, any walk can be written as
\begin{displaymath}
  c = \sum_{P} b_P c_{P}.
\end{displaymath}
The set of this functions with component-wise addition form a group structure, it's called
the \(d\)-chain group and denoted by \(C_d(K)\).

A \(d\)-simplex \(P=\langle p_0, \ldots, p_d \rangle\) is by itself a simplicial complex. 
The orientation of \(P\) suggest a walk given by traversing its maximal faces, that is,
start at the face without \(p_0\), then move to the face without \(p_1\) then to the face
without \(p_2\) and so on. Simbolically,
\begin{displaymath}
  \langle \hat p_0, p_1, \ldots, p_d\rangle + 
  \langle p_0, \hat p_1, \ldots, p_d\rangle 
  + \cdots +
  \langle p_0, \cdots, \hat p_d \rangle
\end{displaymath}

\section{Cohomology}

\begin{thebibliography}{9}
\bibitem{munkres}
  Elements of Algebraic Topology, James Munkres.
\end{thebibliography}

\end{document}
