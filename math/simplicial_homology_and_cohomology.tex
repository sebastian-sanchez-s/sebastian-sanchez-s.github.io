\documentclass{article}

\usepackage{amsmath, amssymb}
\usepackage{tikz}

\newcommand{\R}{\mathbb R}
\newcommand{\Z}{\mathbb Z}

\DeclareMathOperator{\im}{im}

\title{Simplicial Homology and Cohomology\\ A Geometrical Introduction}
\author{Sebastián Sánchez}

\begin{document}

\maketitle

\section{Introduction}

Rigorous definition are based on~\cite{munkres}.

\section{Simplicial Complexes}

A simplicial complex is a generalization of a triangulation. The role of triangles is
played by the simplices. A \(d\)-simplex is collection of \(d+1\) affinely independent points
in \(\R^{d}\), affinely independent means that the relative vectors are linearly independent:
\begin{displaymath}
  a_0, \dots, a_d \quad\textrm{affinely independent}
  \iff
  a_1-a_0, \dots, a_d-a_0 \quad\textrm{linearly independent}.
\end{displaymath}
Naturally, we can associate a geometric realization of a simplex by taking is convex hull.
A simplicial complex \(K\) is a collection of simplices closed under taking subsets and 
closed under the operation of intersection. Explicitly,
\begin{enumerate}
  \item for any \(P\in K\), \(Q\in K\) for any \(Q\subset P\),
  \item for any \(P, S\in K\), \(P\cap S \in K\).
\end{enumerate}

The last definition is actually more general and it's called an abstract simplicial
complex.

\section{Simplicial Homology}

\subsection{Quick and Dirty Introduction to Simplicial Homology}

Let \(K\) be a simplicial complex. Define the oriented indicator function of an
(oriented) simplex \(P\) as follows:
\begin{displaymath}
  c_P(Q) =
  \left\{
  \begin{aligned}
    1 &\quad \textrm{if } Q=P\\
    -1 &\quad \textrm{if } Q=P\text{ as sets but differ in orientation}\\
    0 &\quad \textrm{if} Q\ne P
  \end{aligned}
  \right.,
\end{displaymath}
where an orientation of a simplex is an equivalence class of its ordered vertices
with two elements being in the same class if their ordering differ by an even
permutation.

Denote by \(K^{(d)}\) the simplices of dimension \(d\) of \(K\). A \(d\)-chain is
a linear function from \(K^{(d)}\) to \(\Z\), where \(\Z\).
The set of all \(d\)-chains is denoted as \(C_d(K)\). With point-wise addition
this set becomes a free abelian group whose generators are the oriented indicators.
\begin{displaymath}
  c\in C_d(K) \iff c = \sum_{P\in K^{(d)}} \alpha_P c_P
  \quad \alpha_P\in \Z.
\end{displaymath}

Define the linear map \(\partial_d\colon C_d \to C_{d-1}\) acting as
\begin{displaymath}
  \partial_d [p_0, \ldots, p_d] = \sum_{i=0}^{d} (-1)^{i} [p_0, \ldots, \hat p_i, \ldots, p_d]
\end{displaymath}
where \(\hat p_i\) means that we remove that element from the ordered list and we have identified
the simplices with their indicators.
This map is called the boundary map and satisfies \(\partial_{d} \circ
\partial_{d-1} = 0\). Finally, define the \(d\)-homology group as
\begin{displaymath}
  H_d(K) = \frac{\ker \partial_{d}}{\im \partial_{d+1}}.
\end{displaymath}

\subsection{The Geometry of Simplicial Homology}

\textbf{What is an orientation?}~Consider the filled triangle with vertices
\(A,B\) and \(C\). The usual dry definition (given above) of an orientation is an
ordered list of the vertices: \([A,B,C]\), \([A,C,B]\), etc. with two ordered
list being identified if they differ by an even permutation. I want to convince
you that an orientation is a way to walk around the simplex.

Sticking with the triangle \(ABC\), name by \(a,b\) and \(c\) the
\textit{sides} opposite to vertices \(A,B\) and \(C\), respectively.
These are two ways of describing the same thing (the filled triangle),
but now you can clearly see that an order of \(a,b,c\) gives us a way
of going around the triangle. Let's identify the order \([A,B,C]\)
with \([a,b,c]\) and visualize it as going from face \(a\) to face \(b\) 
and to face \(c\).

You might argue that the order \([A,B,C]\) also gave a clear path of going
around the triangle. I gave you the point, but, is it as clear the path for a
tetrahedron? The nice thing of the above construction is that it works on any
dimension.

\medskip
\textbf{The boundary Map}~ 


\subsection{Applications: Surface Reconstruction}


\section{Simplicial Cohomology}

\begin{thebibliography}{9}
\bibitem{munkres}
  Munkres, James R. Elements of algebraic topology. CRC press, 2018.
\end{thebibliography}

\end{document}
